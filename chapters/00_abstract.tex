% \noindent Machine learning (ML) systems are increasingly deployed in high-stakes domains where reliability is paramount. As these systems transition from research prototypes to real-world decision-makers, their ability to recognize and respond to uncertainty becomes essential. This thesis investigates how uncertainty estimation can enhance the safety and trustworthiness of ML, with a particular focus on selective prediction—a paradigm where models abstain from predicting when confidence is low.

% We begin by showing that a model’s training trajectory contains rich signals that can be leveraged to estimate uncertainty without modifying the architecture or loss function. By ensembling predictions from intermediate checkpoints, we introduce a lightweight, post-hoc abstention mechanism that identifies unreliable predictions. This method applies across classification, regression, and time-series tasks, can be layered onto existing models, and avoids the training cost of deep ensembles while retaining much of their effectiveness. It achieves state-of-the-art performance on multiple selective prediction benchmarks and offers a practical solution for settings where retraining with specific uncertainty-enhancing loss functions is expensive or restricted.

% The utility of this passive, post-hoc approach extends directly to another critical requirement for trustworthy AI: data privacy. Because our method merely observes the training trajectory, it remains fully compatible with formal privacy guarantees like differential privacy (DP). This unique compatibility allows us to investigate a crucial trade-off: how does the enforcement of privacy impact a model's ability to estimate its own uncertainty? We find that many standard methods degrade under DP noise, producing unreliable confidence scores. In contrast, our trajectory-based method remains robust. To fairly evaluate this trade-off, we propose a new framework that isolates the effect of privacy on uncertainty quality, enabling more meaningful comparisons between selective predictors in privacy-sensitive settings.

% This motivates a theoretical study of what fundamentally limits selective prediction performance. We propose a finite-sample decomposition of the selective classification gap—the deviation from the oracle accuracy–coverage curve—and identify five key error sources: Bayes noise, approximation error, ranking error, statistical variability, and a residual term. This decomposition clarifies which levers—such as calibration, model capacity, or additional supervision—can close the gap, and explains why simple post-hoc calibration cannot address ranking imperfections, motivating methods that re-rank predictions based on more reliable uncertainty signals.

% This analysis provides a blueprint for diagnosing and fixing the benign sources of error in a model. It assumes, however, that the model's uncertainty signals, while flawed, are an honest reflection of its internal state. This motivates a deeper investigation into scenarios where uncertainty signals are deliberately corrupted to mislead downstream decision-making. We show that the very mechanisms of ranking and calibration can be adversarially manipulated to increase uncertainty in targeted regions or for specific user groups, enabling covert denial of service while maintaining high predictive performance. These attacks, which directly exploit the sources of error we identified, are hard to detect with standard evaluation. We therefore develop defenses that verify whether abstentions stem from genuine uncertainty, combining calibration audits with verifiable inference to ensure integrity. This highlights a broader lesson: trustworthy ML depends not just on estimating uncertainty well, but also on protecting it from manipulation.

% Together, these contributions chart a path toward more reliable ML by studying how uncertainty can be estimated, evaluated, and safeguarded. The resulting systems not only make accurate predictions—but also know when to say “I do not know”.

\noindent Machine Learning (ML) systems are increasingly deployed in high-stakes domains where reliability is paramount. As these systems transition from research prototypes to real-world decision-makers, their ability to recognize and respond to uncertainty becomes essential. This thesis investigates how uncertainty estimation can enhance the safety and trustworthiness of ML, with a particular focus on selective prediction—a paradigm where models abstain from predicting when confidence is low.

We first show that a model's training trajectory contains rich uncertainty signals that can be exploited without altering its architecture or loss. By ensembling predictions from intermediate checkpoints, we propose a lightweight, post-hoc abstention method that works across tasks, avoids the cost of deep ensembles, and achieves state-of-the-art selective prediction performance. Crucially, this approach is fully compatible with differential privacy (DP), allowing us to study how privacy noise affects uncertainty quality. We find that while many methods degrade under DP, our trajectory-based approach remains robust, and we introduce a framework for isolating the privacy-uncertainty trade-off. Next, we then develop a finite-sample decomposition of the selective classification gap -- the deviation from the oracle accuracy-coverage curve -- identifying five interpretable error sources and clarifying which interventions can close the gap. This explains why calibration alone cannot fix ranking errors, motivating methods that improve uncertainty ordering. Finally, we show that uncertainty signals can be adversarially manipulated to hide errors or deny service while maintaining high accuracy, and we design defenses combining calibration audits with verifiable inference.

Together, these contributions chart a path toward more reliable ML by studying how uncertainty can be estimated, evaluated, and safeguarded. The resulting systems not only make accurate predictions—but also know when to say “I do not know”.