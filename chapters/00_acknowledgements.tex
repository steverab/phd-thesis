\noindent I would first like to express my deepest gratitude to my advisor, \textbf{Nicolas Papernot}, for his unwavering support, insightful guidance, and for fostering an incredibly vibrant and social lab culture. Working under your supervision has been a truly formative experience that has shaped both my academic and personal growth. Nicolas consistently empowered me to follow my curiosity, giving me the freedom to explore research directions I was genuinely passionate about. Your openness to new ideas created an environment where I felt trusted and encouraged to take intellectual risks. What I appreciated most was that our mentorship was a true dialogue—Nicolas was not only generous in sharing his expertise but also genuinely curious to learning from me, which made our collaboration feel deeply mutual and respectful. Beyond research, the lab culture you cultivated stands out as something truly special. Nicolas built a space that was intellectually stimulating and also socially connected—a place where collaboration, support, and friendship were the norm. When I share stories of how our lab operates with students from other institutions, they often remark on how rare and enviable such a community is. I feel incredibly fortunate to have been part of it.

I am also profoundly thankful to my supervisory committee professors—\textbf{Rahul G. Krishnan}, \textbf{Roger Grosse}, \textbf{David Duvenaud}, and \textbf{Zachary C. Lipton}—for their continued high-quality feedback and for challenging me to refine my research directions. Their expertise and generous mentorship have been integral to the progress of my work. I would also like to thank Professor \textbf{Aaron Roth} for serving as my external examiner on this thesis, and Professors \textbf{Chris J. Maddison} and \textbf{Roman Genov} for completing my final examination committee.

I am equally indebted to many current/former \textbf{members of the CleverHans Lab}, in particular \textbf{Mohammad Yaghini}, \textbf{Jonas Guan}, \textbf{Nathalie Dullerud}, \textbf{Sierra Wyllie}, \textbf{Anvith Thudi}, \textbf{Ilia Shumailov}, \textbf{David Glukhov}, \textbf{Nick Jia}, \textbf{Emmy Fang}, \textbf{Adam Dziedzic}, and \textbf{Franziska Boenisch}. Their friendship, thoughtful discussions, and sharp insights played a key role in shaping many of my projects. Beyond my immediate lab, the broader \textbf{Vector Institute community} provided a stimulating research environment, filled with researchers tackling interesting and highly related challenges. It also allowed me to become friends with many more outstanding researchers, including \textbf{Claas Voelcker}, \textbf{Vahid Balazadeh}, \textbf{Viet Nguyen}, \textbf{Andrew Jung}, and \textbf{Aroosa Ijaz}. I will genuinely miss this office space and the incredible people who made it such a great place to work at from the very beginning.

During the past five years, I was lucky to have had the chance to spend most of my summers doing internships in very exciting places. My gratitude goes to the \textbf{Amazon AWS Forecasting team} for hosting me for a total of four internships during my overall studies (one of them during my PhD)—particularly my managers, \textbf{Tim Januschowski} and \textbf{Jan Gasthaus}. Their mentorship and real-world ML deployment challenges were crucial stepping stones to my PhD journey. Similarly, I owe thanks to the \textbf{Google Research team} in Zurich—particularly my managers, \textbf{Nathalie Rauschmayr} and \textbf{Ace Kulshrestha}—whose support and guidance facilitated rich exploration into model cascading. I would also like to extend a heartfelt thanks to \textbf{David Krueger} for hosting me at the University of Cambridge during the summer of 2023, offering me another invaluable opportunity to broaden my research perspectives.

I am sincerely grateful to all my coauthors with whom I worked on a bunch of exciting projects: \textbf{Abhradeep Thakurta}, \textbf{Ace Kulshrestha}, \textbf{Adrian Weller}, \textbf{Adam Dziedzic}, \textbf{Akram Bin Sediq}, \textbf{Ali Shahin Shamsabadi}, \textbf{Angéline Pouget}, \textbf{Anvith Thudi}, \textbf{Armin Ale}, \textbf{Christopher A. Choquette-Choo}, \textbf{Congchao Wang}, \textbf{Emmy Fang}, \textbf{Federico Tombari}, \textbf{Hamza Sokun}, \textbf{Israfil Bahceci}, \textbf{Kimia Hamidieh}, \textbf{Krishnamurthy (Dj) Dvijotham}, \textbf{Mark R. Thomas}, \textbf{Mohammad Yaghini}, \textbf{Muhammad Ahmad Kaleem}, \textbf{Murat A. Erdogdu}, \textbf{Nathalie Rauschmayr}, \textbf{Nicolas Papernot}, \textbf{Olive Franzese}, \textbf{Petra Poklukar}, \textbf{Sean Augenstein}, \textbf{Somesh Jha}, \textbf{Wittawat Jitkrittum}, and \textbf{Xiao Wang}. The work I am discussing in this thesis would not have been possible without you and I am very thankful to have had the chance to work with all of you. 

Special thanks go out to my labmate, flatmate, and dear friend, \textbf{Mohammad Yaghini}. Moving in together at the start of our PhDs was a great choice. I already know that I will miss our joint late-night gaming sessions, the food we cooked together, and our long conversations at the kitchen table. Maybe one day we will even solve world politics together. Thanks for celebrating my victories with me, but also, and arguably even more importantly, for being there for me when I struggled at work and in my personal life. Similar thanks go out to my longtime friend, \textbf{Lukas Prantl}, for making sure I stay grounded in what really matters by ensuring that I continue to play Age of Empires II with him on a (semi-)regular basis. 

Lastly, I am profoundly grateful to my immediate family—my dad \textbf{Martin}, my mom \textbf{Verena}, and my sister \textbf{Monika}—for their unconditional kindness and support. Your belief in me has fueled my determination every step of the way. More importantly, you have shown me what a loving and caring family looks like, and I hope that one day I will be able to build the same kind of warm, supportive, and joyful home for the people I love. 
